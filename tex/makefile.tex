\documentclass{article}
\usepackage{ctex}
\begin{document}

make命令執行时,需要一个Makefile文件以告诉make命令需要怎么样的去编译和链接程序。
基本规则:
\begin{enumerate}
	\item  如果这个工程没有编译过,则所有C文件都要编译并被链接。
	\item  如果这个工程的某几个C文件被修改,则只编译被修改的C文件,并链接目标程序。
	\item  如果这个工程的头文件被修改,则需要编译引用了这个头文件的C文件,并链接目标程序。
\end{enumerate}
Makefile规则:
\begin{verbatim}
target ... : prerequisites ...
             command
             ...
             ...			
\end{verbatim}
target是一个目标文件,可以是Object file也可以是可执行文件,还可以是一个标签(Label)。
prerequisites是要生成target所需要的文件或目标。
command是make需要执行的命令(任意的Shell命令)
也就是,target这一个或多个的目标文件依赖于prerequisites中的文件,其生成规则定义在command中。
再即是,prerequisites中如果有一个以上的文件比target文件新,则command命令就会执行。
定义好依赖关系后,下一行定义了如何生成目标文件的操作系统命令,一定要以一个Tab键作为开头。make并不关心命令是怎么工作的,它只是执行所定义的命令。
clean不是一个文件,而是一个动作的名字,像C语言的label一样,其冒号后面没有文件,则make就不会自动去找文件的依赖性,也就不会自动执行其后所定义的命令。要执行其后的命令,就要在make命令后显式的指出这个label的名字。这样就可以在一个Makefile文件中定义不用的编译或是和编译无关的命令,比如程序的打包、备份等等。

\section{工作原理}
默认的方式下(只输入make),那么:
\begin{enumerate}
\item  make在当前目录下查找Makefile或makefile文件。
\item  如果找到,就会找到文件中的第一个目标文件,并把这个文件作为最终的目标文件。
\item  如果目标文件不存在,或是目标文件所依赖的文件的修改时间比目标文件新,则执行后面定义的命令来生成目标文件。
\item  如果目标文件所依赖的.o文件存在,make会在当前文件中查找.o文件所依赖的文件,如果找到则再根据那一个规则生成.o文件。
\item  最后一定是源代码文件,于是会make生成.o文件,再用.o文件执行make,生成目标文件。
\end{enumerate}
这就是make的依赖性,一层一层的去找文件的依赖关系,直到最后编译出第一个目标文件。
注意make只解决依赖关系,而不管所定义的命令的错误或编译是否成功。
\section{使用变量}
如声明一个变量,OBJECTS,则在引用时可以
\begin{verbatim}
OBJECTS = main.o kbd.o command.o display.o \
          insert.o
edit: $(OBJECTS)
	   cc -c edit $(OBJECTS)
.....
clean:
	   rm edit $(OBJECTS)
\end{verbatim}
\section{让make自动推导}
只要make看到一个.o文件,就会把相应的.c文件加到.o文件的依赖中。例如会自动把main.c加到main.o的依赖文件中。于是类似上面的Makefile又可以省略.o和.c文件同名的。
\section{另类的Makefile}

\section{伪目标}
.PHONY: clean表示clean是一个伪目标,不成文的规定是clean从来都放在Makefile文件的最后。
\section{细节}
Makefile里主要包含了5个东西:显式规则,隐晦规则,变量定义,文件指示和注释。
\begin{enumerate}
\item  显式规则:显式规则说明了如何生成一个或多个目标文件文件。由Makefile的书写者显式的指出要生成的文件,文件的依赖文件和生成的命令。
\item  隐晦规则:由于make有自动推导功能,所以隐晦规则可以允许比较粗糙简略的书写Makefile,这是make所支持的。
\item  变量的定义:在Makefile中要定义一系列的变量,变量一般都是字符串,类似C语言的宏,当Makefile被执行时,其中的变量都会被扩展到相应的引用位置上。
\item  文件指示:包括三部分,一个是在一个Makefile中引用另一个Makefile,就像C语言的\#include;另一个是根据某些情况指定Makefile中的有效部分,就像C语言的预编译\#if;另一个是定义一个多行的命令。
\item  注释:Makefile只有行注释,使用\#作为注释标记。
\end{enumerate}
\subsection{文件名}
默认查找顺序:GNUmakefile,makefile, Makefile用的最多的是Makefile。如果要指定makefile文件,使用-f选项。
\subsection{引用}
使用include关键词把别的Makefile包含进来。
\begin{verbatim}
include <filename>
\end{verbatim}
filename可以是当前操作系统Shell的文件模式(可以包含路径和通配符)。
\subsection{环境变量MAKEFILES}
如果当前环境变量中定义了MAKEFILES,那么make会把这个变量中的值做一个类似于include的动作。这个变量的值是其他的Makefile,用空格分隔。但是从这个环境变量中引入的Makefile的目标不回起作用。如果环境变量中的文件发现错误,make也不会理会。
\subsection{工作方式}
\begin{enumerate}
\item  读入所有的Makefile
\item  读入被include的其他Makefile
\item  初始化文件中的变量
\item  推导隐晦规则,并分析所有规则
\item  为所有的目标文件创建依赖关系链
\item  根据依赖关系,决定哪些目标需要重新生成
\item  执行生成命令
\end{enumerate}
\subsection{vpath}
设置文件搜索路径
VPATH=src:../headers
即定义了两个路径
也可以使用vpath关键字(全是小写的,不是环境变量),可以指定不同的文件在不同的搜索目录中。使用方法有三种:
\begin{enumerate}
\item  vpath <pattern> <directories>
为符合模式的文件指定搜索目录
\item  vpath <pattern>
清除符合模式的文件搜索目录
\item  vpath
清除所有已经被设置好了的文件搜索目录
\end{enumerate}
<pattern>需要包含\%字符,表示通配符(零个或多个字符)
\subsection{伪目标}
以make clean为例,clean并不是一个目标文件,而只是一个标签。由于伪目标不是文件,所以make无法生成依赖关系和决定是否要执行。只有通过显式的指明这个目标才能让其生效。
\end{document}